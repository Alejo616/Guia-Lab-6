% !TEX root = ../notes_template.tex
\chapter{Medidas e incertidumbres}\label{p1}

El objetivo de esta pr\'actica es aprender a hallar la ecuaci\'on matem\'atica 
de una ley f\'isica mediante m\'etodos gr\'aficos. En distintas \'areas de la 
f\'isica, en particular en f\'isica estad\'istica, muchos de los fen\'omenos 
pueden ser descritos con ecuaciones exponenciales tipo
\begin{equation*}
 y = A e^{Bx}.
\end{equation*}
Un ejemplo claro de ello son las funciones de distribuci\'on de los fermiones y 
los bosones:
%%%%%
\begin{eqnarray*}
 \text{Fermi-Dirac: } \quad f &=&\frac{1}{1+e^{-[\frac{E-F}{kT}]}}, \\
 \text{Maxwell-Boltzman: } \quad f &=& c e^{-[\frac{E}{kT}]}, \\
 \text{Bose-Einstein: } \quad f &=&\frac{1}{1-e^{-[\frac{E-F}{kT}]}}.
\end{eqnarray*}
%%%%%
Estas ecuaciones determinan la probabilidad de distribuci\'on de las 
part\'iculas por las energ\'ias en funci\'on de su temperatura. Por ejemplo, 
las concentraci\'on de $n$ electrones libres en un s\'olido sigue una 
distribuci\'on de Maxwell-Boltzman y de pende la tempreratura $T$:
%%%%%%
\begin{equation*}
 n = N_o e^{-\left[\frac{E_g}{k_B T}\right]},
\end{equation*}
%%%%%%
d\'onde $N_o$ es el n\'umero de electrones efectivos, $E_g$ la barrera 
energ\'etica, $k_B$ la constante de Boltzman y $T$ la temperatura. 
%%%%%%
En esta pr\'actica de laboratorio buscamos experimentalmente la distribuci\'on 
de unas arandelas sobre dos l\'ineas fijas y su dependecia con el n\'umero 
total de arandelas disponibles en la tabla. la ecuaci\'on buscada es de tipo 
exponencial y la tarea es hallar esta ecuauci\'on por medio de un m\'etodo 
gr\'afico. Cabe resaltar que se usar\'an dos m\'etodos diferentes para hallar 
la misma ecuaci\'on.
